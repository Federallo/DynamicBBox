% setting document languate
\documentclass[italian]{article}

% setting italian font encoding
\usepackage[utf8]{inputenc}
\usepackage[T1]{fontenc}
\usepackage{babel}

\begin{document}
\section{Introduzione}
Il seguente progetto consiste nell'aggiornare i bounding box generati inizialmente dall'algoritmo della repository blensor\_analysis.\\
Per poter inizializzare il cluster all'interno della bounding box, è stato utilizzato un algoritmo di DBSCAN custom, il quale prende come argomenti la bounding box dello scan corrente e la point cloud dello scan successivo.\\
Una volta determinati i punti interni alla bounding box, seleziona il cluster con più punti al suo interno, il quale rappresenta l'oggetto relativo alla bounding box, e ne crea un nuovo bounding box.\\
In caso ci fossero dei punti esterni dai bounding box, questi vengono analizzati dall'algoritmo della repository blensor\_analysis e, in caso venissero indivduate nuovi bounding box, nuovi aggiunti alla lista di bounding box da aggiornare.\\
\section{Descrizione dei file}
Nella seguente sezione vengono trattate nel dettaglio i singoli file del progetto.
\subsection{blensoranalysis}
Il file blensoranalysis presenta metodi che servono da tramite tra DynamicBBox e la repository blensor\_analysis.
\subsubsection{generateBoundingBox}
%TODO aggiungi lo scan e la filtered value
Il seguente metodo viene utilizzato per generare le prime bounding boxes a primo avvio del programma e vengono mostrate anche in output tramite liberia open3d.
\subsubsection{generateBB}
In generateBB le bounding box vengono generate tramite pointcloud data in input e vengono, in un momento successivo, aggiunte alla lista di bounding box aggiornate.
\subsection{main}
Il file main.py è il file principale del progetto, il quale inizializza le prime bounding box utilizzando metodi presenti nella repository blensor\_analysis utilizzando la libreria blensoranalysis che funge da tramite tra DynamicBBox e blensor\_analysis.\\
I prossimi scan vengono inizializzati tramite ciclo for
\end{document}
