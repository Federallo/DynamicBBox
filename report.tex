% setting document languate
\documentclass[italian]{report}

% setting italian font encoding
\usepackage[utf8]{inputenc}
\usepackage[T1]{fontenc}
\usepackage{babel}
\usepackage{microtype}
\usepackage{ragged2e}
\usepackage{pgfplots}
\usepackage{makeidx}
\usepackage{sidecap}
\usepackage{graphicx}
\usepackage{subcaption}
\usepackage{caption}
\usepackage{float}
\usepackage{hyperref}
\usepackage{helvet}
\graphicspath{ {./images/} }
\captionsetup{font=footnotesize}
\makeindex
\hypersetup{
    colorlinks,
    citecolor=black,
    filecolor=black,
    linkcolor=black,
    urlcolor=black
}

\begin{document}
\begin{titlepage}
	\centering
	\includegraphics[width=0.2\textwidth]{logo}\par\vspace{1cm}
	{\scshape\Large Università degli Studi di Firenze \par}
	\vspace{0.5cm}
	{\scshape\large Scuola di Ingegneria - Dipartimento di Ingegneria dell'Informazione \par}
	\vspace{1.5cm}
	{\textbf{Tesi di Laurea Triennale in Ingegneria Informatica} \par}
	\vspace{2cm}

	{\LARGE\bfseries Clustering sequenziale di point cloud per il rintracciamento di veicoli \par}

	\vspace{2cm}
	{\textit{Candidato} \par}
	{\itshape Andrea Aligi Assioma\par}
	\vspace{1cm}
	{\textit{Relatore} \par}
	{\itshape Prof. Giorgio Battistelli\par}
\end{titlepage}

\justifying

\tableofcontents
\newpage
\chapter*{Introduzione}
\addcontentsline{toc}{chapter}{Introduzione}
Negli ultimi anni la rapida evoluzione delle tecnologie dei sensori e l'aumento della potenza di calcolo dei computer ha portato a una crescita esponenziale dell'uso di dati tridimensionali in diversi campi, tra cui la robotica, la  guida autonoma e la computer vision.\\
Le Point Cloud rappresentano un esempio di strumento per la rappresentazione e l'analisi di ambienti tridimensionali. Generate da sensori come LiDAR o telecamere stereoscopiche, le point cloud in movimento rappresentano un problema per la rilevazione e il tracking di oggetti in quanto richiedono un'elaborazione in tempo reale. Le bounding box sono un valido strumento per la rappresentazione di tali oggetti. Oltre a rispettare la forma dell'oggetto, devono essere anche in grado di adattarsi a cambiamenti di forma e dimensione, o a cambiamenti temporali.\\
\\
Il progetto si pone l'obiettivo di sviluppare un algoritmo di clustering per point clouds in movimento, in modo da generare bounding box precise e adattabili a cambiamenti di forma e dimensione.\\
Il corretto aggiornamento delle bounding box rappresenta un passo cruciale per lo scopo del progetto, in quanto permette di avere una rappresentazione precisa degli oggetti in movimento.\\
\\
La tesi è strutturata come segue: la prima parte sarà dedicata ad approfondire le principali tecnologie utlizzate per lo sviluppo del progetto, la seconda invece sarà dedicata agli algoritmi implementati. La terza parte sarà dedicata alla descrizione di tutti i passaggi del programma e nella quarta e quinta verranno mostrati gli esiti ottenuti e verrano tratte delle conclusioni.
\newpage

\chapter{Tecnologie utilizzate}
Nella seguente sezione saranno elencate le tecnologie utilizzate per lo sviluppo del progetto. Alcune di esse, non facendo direttamente parte dello scopo del progetto, verranno passate in rassegna più rapidamente.
\section{LiDAR}
LiDAR, o Light Detection and Ranging, rappresenta uno strumeto di telerilevamento che determina la distanza di un oggetto o superfice tramite l'utilizzo di impulsi laser.\\
Il risultato di tale processo è una pointcloud \cite{LiDAR}.\\
\begin{figure}[H]
	\centering
	\includegraphics[width=0.9\textwidth]{LiDAR}
	\footnotesize
	\caption{Point cloud dettagliata di una strada ottenuta tramite LiDAR}
\end{figure}
\section{Datasets}
Affinché si avesse una generazione di pointcloud affidabile a partire dai dataset, è stato necessario l'utilizzo della tecnologia \textit{Arogverse} per l'acquisizione di scenari e modelli e di \textit{Blensor} per simulare scansioni LiDAR.
\subsection{Argoverse}
Argoverse è un dataset open-source di mappe, sviluppato dall'Università Carnegie Mellon e dal Georgia Institute of Tecnology negli Stati Uniti, il quale include metadati utili per lo sviluppo di sistemi a guida autonoma.\\
Tali dati sono stati raccolti a Miami e a Pittsburgh, due città caratterizzate da due comportamenti di guida differenti. Nel progetto è stato analizzato il traffico presente in un incrocio di Pittsburgh (specificando per ciascun veicolo ID e posizione).\\
\begin{figure}[H]
	\centering
	\includegraphics[width=0.9\textwidth]{Pittsburgh}
	\footnotesize
	\caption{Mappa di un incrocio di Pittsburgh}
\end{figure}\\
Nonostante le informazioni siano state raccolte attraverso veicoli con telecamere e sensori LiDAR, questi ultimi non influenzano assolutamente lo scopo del progetto.
\subsection{BlenSor}
BlenSor è un package dell'applicazione Blender che permette di simulare scansioni LiDAR. Sviluppato dall'Università di Salisburgo, ha come principale vantaggio quello di creare scenari che, nel caso della guida autonoma, nella realtà non sarebbero realizzabili o sarebbero troppo costosi da riprodurre (come per esempio una sitazione di incidente).\\
Nel nostro caso è risultato fondamentale per l'utilizzo di algoritmi, di norma presenti in sensori fisici, senza possederli realmente.\\
\begin{figure}[H]
	\centering
	\includegraphics[width=0.9\textwidth]{Blensor}
	\footnotesize
	\caption{Scansione LiDAR simulata tramite Blensor}
\end{figure}\\
La simulazione tramite Blensor genera un file .csv per ogni istante e contiene le coordinate dei punti della pointcloud (senza rumore e filtrati, dove quest'ulitmo termine indica che viene rimosso lo scan del terreno).
\section{Open3D}
Open3D è una libreria open-source che dispone di funzionalità grafiche per programmi Python e C++\cite{Open3D}. Le principali features sono:
\begin{itemize}
	\item la possibilità di creare strutture dati 3D
	\item la presenza di algoritmi di data processing in 3D utilizzabili 
	\item lo sfruttamento della GPU per velocizzare operazioni più laboriose.
\end{itemize}
Per lo sviluppo del progetto sono stati sfruttati soprattutto i metodi di clusterizzazione delle pointcloud, di ricerca dei punti di una pointcloud vicini e di creazione delle bounding box a partire da cluster, insieme alla visualizzazione grafica.
\subsection{Pointcloud}
La pointcloud è una struttura dati che rappresenta un insieme di punti nello spazio 3D. Viene istanziata tramite \textit{geometry.PointCloud()} e i punti (di coordinate (x,y,z)) vengono aggiunti come \textit{utlityVector3dVector(list)}\\
\begin{figure}[H]
	\centering
	\includegraphics[width=0.7\textwidth]{Pointcloud}
	\footnotesize
	\caption{Esempio di pointcloud filtrata, senza noise e sottocampionata}
\end{figure}\\
\subsection{Sottocampionamento}
Essendo il costo di campionamento delle pointcloud molto elevato a causa dell'alta densità di punti, è stato necessario sottocampionarle per avere una approssimazione accettabile in tempi ridotti. Attraverso \textit{pointcloud.voxeldownsample(size)}  viene sottocampionata la pointcloud attraverso cubi di lato \textit{size}. Questi ultimi dividono lo spazio 3D in griglie e se ciascun cubo presenta una serie di punti, ne verrà mantenuto uno solo.
\begin{figure}[H]
	\centering
	\includegraphics[width=0.7\textwidth]{Voxel}
	\footnotesize
	\caption{Esempio di processo di sottocampionamento della pointcloud}
\end{figure}
\subsection{Clustering}
Open3D dispone di un algoritmo di clustering, \textit{pointcloud.cluster\_dbscan(eps, min\_points, print\_progress)}, che permette di clusterizzare i punti della pointcloud.\\
E' un metodo di raggruppamento di punti che si basa sulla distanza tra di essi e su un numero di punti minimo prestabilito che sta a indicare la presenza di un cluster.\\
Ovviamente questo metodo viene sfruttato solo in due situazioni nel progetto: per la prima creazione delle bounding box e per la clusterizzazione di punti esterni alle bounding box.
\subsection{K-D Tree}
Il K-D Tree (o K-Dimensional Tree) è una struttura dati che permette di effettuare ricerche in spazi multidimensionali in maniera efficiente, sfruttando il concetto di albero binario. Open3D dispone di una classe \textit{geometry.KDTreeFlann()} che permette di creare un K-D Tree a partire da una pointcloud.\\
Rispetto a una normale ricerca sequenziale, il K-D Tree riduce i tempi di ricerca da $O(n)$ a $O(log(n))$. Ciò è dovuto soprattutto al fatto che vengono eliminati quei sotto-alberi di punti che si trovano a una distanza superiore a quella data. Un limite del K-D Tree è che necessita spazi di dimensione bassa (e quindi per uno spazio tridimensionale va più che bene) per poter effettuare le ricerche in tempi bassi.
\begin{figure}[H]
	\centering
	\includegraphics[width=0.5\textwidth]{3dtree}
	\footnotesize
	\caption{Suddisione di uno spazio 3D in un K-D Tree}
\end{figure}\\
In questo progetto è stato sfruttato nel clustering personalizzato per trovare i punti vicini a un punto dato. Attraverso il metodo \textit{KDTreeFlann().search\_radius\_vector\_3d(point, radius)}, vengono restituiti gli indici dei punti della pointcloud che si trovano a una distanza minore di \textit{radius} dal punto dato.

\subsection{Bounding Box}
Le Bounding Box sono rettangoli (o parallelepipedi in 3D) che racchiudono un insieme di punti della pointcloud.\\
Esistono tre categorie di bounding box:
\begin{itemize}
	\item Axis-Aligned Bounding Box (AABB): rappresenta una bounding box allineata con gli assi cartesiani. Dei tre è il più veloce da calcolare, ma può ricoprire più area del normale.
	\item Obiented Bounding Box (OBB): è una bounding box orientata in base alla forma della pointcloud. E' più accurata dell'AABB, ma richiede più tempo per essere calcolata.
	\item Minimum Bounding Box (MBB): individua la bounding box con la minima area che racchiude l'insieme di punti della pointcloud. Soprattutto in dimensioni superiori a due, richiede molto tempo per essere calcolata.
\end{itemize}
Nel nostro caso è stata utilizzata la bounding box di tipo OBB, in quanto la più accurata per lo scopo del progetto.
\begin{figure}[H]
	\centering
	\includegraphics[width=0.7\textwidth]{bbox}
	\footnotesize
	\caption{Esempio di bounding box orientate}
\end{figure}\\
Open3D offre la possibilità di creare le bounding box a partire dalla pointcloud utilizzando il metodo \textit{pointcloud.get\_oriented\_bounding\_box()}, il quale ritorna un oggetto
\textit{geometry.OrientedBoundingBox}.\\
Le \textit{OrientedBoundingBox} possiedono l'attributo per assegnargli un colore, e cioè \textit{obb.color = [r,g,b]}.
\subsection{Punti interni alla bounding box}
Siccome all'interno del programma sarà necessario creare cluster nel modo più rapido possibile, la funzione\\ \textit{OrientedBoundingBox.get\_point\_indices\_within\_bounding\_box(pointcloud.points)} ha permesso di individuare gli indici dei punti interni alla bounding box in determinato un istante di tempo, da cui poi creare il nuovo cluster e aggiornare la sua posizione.\\ 
Essenzialmente analizza un insieme punti della pointcloud dati come argomento alla funzione (in questo caso \textit{pointcloud.points}) e ritorna un array di indici. Un punto viene considerato interno alla bounding box se, proiettato sui tre assi, risulta compreso tra i valori minimi e massimi determinati attraverso l'attributo \textit{extent} della OBB.\\
Questo metodo è ottimizzato per pointcloud di grandi dimensioni e garantisce un'ottima accuratezza nell'individuare i punti.
\subsection{Centro, estensione e orientazione della bounding box}
L'oggetto \textit{geometry.OrientedBoundingBox} dispone di tre attributi fondamentali:
\begin{itemize}
	\item center: rappresenta il centroide della bounding box.
	\item extent: rappresenta l'estensione della bounding box. Sono tre valori che rappresentano l'altezza, la larghezza e la profondità della bounding box e rispettano l'orientamento di quest'ultima.
	\item R: rappresenta una matrice 3x3 che definisce l'orientamento della bounding box nello spazio tridimensionale.
\end{itemize}
Le seguenti informazioni hanno permesso di espandere la bounding box e includere nuovi punti alla porzione di pointcloud che si sta prendendo in analisi.
\begin{figure}[H]
	\centering
	\begin{subfigure}{0.45\textwidth}
		\includegraphics[width=\textwidth]{notExpandedBox}
	\end{subfigure}
	\begin{subfigure}{0.45\textwidth}
		\includegraphics[width=\textwidth]{expandedBox}
	\end{subfigure}
	\footnotesize
	\caption{Espansione della bounding box}
\end{figure}

\subsection{Sfera}
Uno dei due metodi di clusterizzazione implementati prevede la creazione di una sfera a partire dai dati forniti da una bounding box. Open3D permette di istanziare un oggetto \textit{open3d.geometry.TriangleMesh} che crea e manipola geometrie 3D a mesh triangolari. Una mesh triangolare è un tipo di mesh poligonale in cui la superficie di un oggetto 3D è composta da facce triangolari.\\
Il metodo \textit{open3d.geometry.TriangleMesh.create\_sphere(radius)} crea una sfera di raggio \textit{radius} e centrata nel centroide della bounding box attraverso il metodo \textit{translate(translation)}.\\
Essendo una sfera colorata, impedisce la visualizzazione dei punti interni: è stato necessario quindi trasformarla in un wireframe. Istanziando un oggetto\\ \textit{open3d.geometry.LineSet.create\_from\_triangle\_mesh(mesh)}, quest'ultimo permette di estrarre dalla mesh della sfera i lati dei triangoli che la compongono, così da permettere la visualizzazione dei punti interni.\\
\begin{figure}[H]
	\centering
	\begin{subfigure}{0.45\textwidth}
		\includegraphics[width=\textwidth]{notExpandedBox}
	\end{subfigure}
	\begin{subfigure}{0.45\textwidth}
		\includegraphics[width=\textwidth]{Sphere}
	\end{subfigure}
	\footnotesize
	\caption{Creazione di una sfera a partire dalla bounding box}
\end{figure}
\subsection{Draw}
Open3D offre la possibilità di visualizzare tramite interfaccia grafica le pointcloud e le bounding box usando il metodo \textit{open3d.visualization.draw()} che permette anche di navigare liberamente all'interno della finestra, di zoomare e/o ruotare gli oggetti.\\
Sono presenti inoltre parametri che permettono una visualizzazione più personalizzata come
\begin{itemize}
	\item mostrare o nascondere piani arbitrari oppure
	\item mostrare o nascondere gli assi di riferimento
\end{itemize}

\begin{figure}[H]
	\centering
	\includegraphics[width=0.7\textwidth]{o3dInterface}
	\footnotesize
	\caption{Esempio di interfaccia grafica di Open3D, con tabella di menu a destra}
\end{figure}\\
Le rappresentazioni tramite interfaccia grafica sono state sfruttate soprattutto per mostrare i passaggi che effettua il programma.

\section{blensor\_analysis}
blensor\_analysis rappresenta la repository di partenza da cui è stato sviluppato il progetto.Nel codice di questa repository vengono prima importate le pointcloud e poi vengono clusterizzate e generate le bounding box, tenendo traccia della traitteoria dei vari oggetti in movimento. Per lo scopo del progetto sono stati sfruttati solo alcuni aspetti, tra cui:
\begin{itemize}
	\item BLENSOR: rappresenta la directory che contiene tutti i file .csv delle pointcloud generate da una quantità sensori che arriva al massimo a 5.
	\item dataset\_loader: rappresenta il modulo che genera le pointcloud a partire dai file .csv e le restituisce come lista di punti.
	\item pointcloud: è il modulo che si occupa di generare le effettive pointcloud sottocampionate e filtrate, clusterizzandole e generando le relative bounding box.
\end{itemize}

\newpage
\chapter{Alogirtmi}
Nella seguente sezione verranno trattati i principali algoritmi implementati per il progetto.
\section{Clustering}
Il clustering è un metodo di Unsupervised Learning che raggruppa un insieme di oggetti in classi simili. Tali classi possono essere di tipo morfologico, cromatico o basata sulle densità degli oggetti.\\
E' usato principalmente nell'esplorazione dei dati per ridurne la complessità di ricerca, per raggrupparli in maniera efficiente oppure per comprimerli.\\
L'algoritmo di clustering implementato nel progetto è il DBSCAN.
\subsection{DBSCAN}
L'algoritmo DBSCAN divide i punti della pointcloud in batch o gruppi in base, in questo caso, alla densità.\\
Prima di descrivere l'argoritmo nel dettaglio, è necessario definire le seguenti terminologie:
\begin{itemize}
	\item Core point: un punto è un core point se ha almeno un numero minimo di punti vicini, il cui valore è chiamato \textit{minPts}, entro una certa distanza, chiamata \textit{epsilon} o \textit{eps}.
	\item Border point: un punto è un border point se non è un core point, ma è raggiungibile da un core point.
	\item Noise point: un punto è un noise point se non è nè un core point nè un border point.
\end{itemize}
\begin{figure}[H]
	\centering
	\includegraphics[width=0.5\textwidth]{clustering}
	\footnotesize
	\caption{Illustrazione di core point, border point e noise point}
\end{figure}\\
I passi principali dell'algoritmo sono:
\begin{itemize}
	\item individua i core points.
	\item per ogni core point che non è stato assegnato a un cluster, ne crea uno nuovo, aggiungendo tutti i punti raggiungibili.
	\item cerca ricorsivamente i punti \textit{density-connected} e li aggiunge allo stesso cluster come core points. Per \textit{density-connected} si intende che due punti sono collegati se esiste un core point che li collega.		Quindi se un punto \textit{b} è raggiungibile da un punto \textit{a} (e quindi che si trova a una distanza massima \textit{eps}), e un punto \textit{c} è raggiungibile da \textit{b}, allora \textit{c} è raggiungibile da \textit{a}.
	\item ripete i passi precedenti fino a quando tutti i punti sono assegnati a un cluster. Quelli che non appartengono neanche a uno, sono considerati noise points.
\end{itemize}
\begin{figure}[H]
	\centering
	\includegraphics[width=0.7\textwidth]{dbscan}
	\footnotesize
	\caption{Clustering di una pointcloud tramite DBSCAN}	
\end{figure}\\
Rappresenta un algoritmo molto resistente a noise e può individuare cluster di forma arbitraria. La complessità dell'algoritmo è quadratica ($O(n^2)$)\cite{DBSCAN}.\\
Come accennato in precedenza, nel caso del progetto viene utilizzato solo per clusterizzare la pointcloud all'avvio del programma e ogni volta che ci sono nuove pointcloud da analizzare e non hanno una bounding box associata.
\subsection{DBSCAN personalizzato}
L'algoritmo di DBSCAN personalizzato, rispetto a quello originale, prende in considerazione dei punti già clusterizzati come punto di partenza e ne completa la creazione.\\
Quindi, partendo da delle bounding box già inizializzate (che vengono distinte dal colore verde) e dalla pointcloud di uno scan successivo, l'algoritmo compie i seguenti passaggi:
\begin{itemize}
	\item data una bounding box, individua gli indici dei punti interni sfruttando la funzione \textit{get\_point\_indices\_within\_bounding\_box()} che chiameremo \textbf{punti core}.
	\item a seconda del metodo di clusterizzazione scelto espande/ trasforma la bounding box in due modi differenti.
		\begin{itemize}
			\item dbscan\_expand\_bbox: espande la bounding box creandone una nova aumentata di un valore prestabilito e riutilizza la funzione citata precedentemente per ottenere gli indici dei nuovi punti.
			\item dbscan\_sphere\_bound: genera una sfera con centro nel centroide della bounding box e di raggio pari alla dimensione maggiore della bounding box per un fattore di espansione dato in input, diviso due 					(la divisione serve a indicare il raggio). Successivamente individua gli indici dei punti interni alla sfera.
		\end{itemize}
	\item una volta deifiniti questi due set di indici, istanzia un array di label delle stesse dimensioni del numero di indici dei punti generati dal file \textit{dbscan\_expand\_bbox} o \textit{dbscan\_sphere\_bound} e vengono settati a -1, per indicare che sono noise. Successivamente setta a 1 tutti quegli indici che sono presenti nei \textbf{punti core}. Inoltre, viene istanziato un oggetto \textit{KDTreeFlann()}, cioè un K-D Tree a partire dalla pointcloud.
	\item a questo punto viene applicato l'effettivo algoritmo dbscan, con la differenza che parte dai punti che hanno label 1 (cioè che sono già considerati appartenenti a un cluster) e cerca e clusterizza quelli che hanno label -1. La ricerca viene eseguita tramite il K-D Tree che sfrutta la funzione la funzione \textit{search\_radius\_vector\_3d()} dalla quale si ottiene un array di indici dei punti che si trovano a una distanza prestabilita da uno dato in input.
	\item la lista di label viene sfruttata per creare un solo cluster, dal quale si genererà la bounding box aggiornata di colore rosso per indicare che si sta tenendo traccia di quell'oggetto.
	\item se in uno scan successivo il numero minimo di punti per formare un cluster è troppo basso, l'oggetto non viene più seguito e viene cancellata la bounding box associata.
\end{itemize}
Con questo approccio si pressuppone che si possa clusterizzare le pointcloud in maniera più efficace in quanto si parte con luster già avviati.\\
\section{Intersezione di bounding box}
Avendo la possibilità di clusterizzare pointcloud esterne alle bounding box, può capitare che in un istante precedente si abbia una bounding box che contiene solo parte di un oggetto (in quanto il veicolo sta ancora entrando nel range dei sensori) e che in un istante successivo si abbia una nuova bounding box che contiene la parte restante, o l'oggetto completo, ma che comunque si sovrappone alla precedente.\\
\begin{figure}[H]
	\centering
	\includegraphics[width=0.7\textwidth]{Overlapping}
	\footnotesize
	\caption{Situazione di overlapping tra bounding box}
\end{figure}\\
Per ovviare a ciò è risultato necessario implementare un algoritmo che confronta le bounding box a coppie. L'operazione di verifica di collisione tra bounding box è possibile analizzando la sovrapposizione (ovrelap) minima e massima per ciascun asse.\\
Dati come assi di riferimento i tre cartesiani:
\begin{itemize}
	\item l'overlap minimo viene calcolato tramite valore massimo di ciascuna delle coordinate interessate delle due bounding box prese in analisi che hanno valore minimo.
	\item l'overlap massimo viene calcolato tramite valore minimo di ciascuna delle coordinate interessate delle due bounding box prese in analisi che hanno valore massimo.
\end{itemize}
Se gli overlap massimi di ciascun asse sono maggiori degli overlap minimi, allora siginfica che le bounding box si sovrappongono e quindi viene rimossa quella che ha volume minore. Questo valore è ottenuto dalla funzione \textit{get\_volume()} che calcola il volume della bounding box come prodotto delle lunghezze dei tre lati.\\

\chapter{Progetto}
Lo scopo del progetto consiste nell'aggiornare le bounding box generate inizialmente dall'algoritmo della repository \textit{blensor\_analysis} in modo da ottimizzare la clusterizzazione delle pointcloud interne alle bounding box. A seconda delle necessità, la clusterizzazione in istanti successivi viene effettuata o all'interno di bounding box espanse oppure all'interno di sfere.\\
I principali parametri che ne determineranno le prestazioni sono il tempo di esecuzione e la precisione delle bounding box.\\ 
Il programma è strutturato nella seguente maniera:
\begin{itemize}
	\item importazione della prima scansione, generazione delle pointcloud e creazione delle prime bounding box
	\item aggiornamento delle bounding box a ogni scansione successiva e visualizzazione grafica di ogni passaggio
\end{itemize}
\section{Hardware e librerie utilizzati}
Le principali librerie e il package utilizzato per lo sviluppo dell'applicativo python sono:
\begin{itemize}
	\item blensor\_analysis per la generazione delle prime bounding box e per la creazione delle pointcloud utili agli istanti successivi
	\item open3D
	\item numpy per la gestione di array
\end{itemize}
Come hardware invece è stato utilizzato un computer con processore i5-10600k e 16GB di RAM. Come IDE è stata utilizzata neovim.
\section{Note sul progetto}
Come detto in precedenza, si sta analizzando un incrocio di Pittsburgh, nel quale sono stati posizionati 5 sensori ad altezza 2.5 metri da terra. La frequenza di scansione, in base alla velocità media dei veicoli, è di 10Hz.
\begin{figure}[H]
	\centering
	\includegraphics[width=0.7\textwidth]{junction_sensors}
	\footnotesize
	\caption{Incorcio di Pittsburgh con i 5 sensori (in nero)}
\end{figure}\\
\section{importazione della prima scansione, generazione delle pointcloud e creazione delle prime bounding box}
In questa fase iniziale del progetto viene sfruttato il modulo \textit{blensoranalysis} che invoca i principali metodi della repository \textit{blensor\_analysis} per importare la pointcloud iniziale dal rispettivo file \textit{.csv}, generando anche i cluster e le bounding box, e rendendendoli visibili tramite finestra generata da Open3D.\\
In questo caso è stato deciso che le pointcloud venissero preventivamente filtrate, rimuovendo gli scan del terreno, e senza rumore, in quanto non fanno parte dello scopo del progetto.
\begin{figure}[H]
	\centering
	\includegraphics[width=0.7\textwidth]{blensor_analysis}
	\footnotesize
	\caption{Pointcloud e bounding box generate da blensor\_analysis}
\end{figure}\\
\section{aggiornamento delle bounding box a ogni scansione successiva}
A questo punto vengono coinvolti i moduli \textit{bbtracking.py}, \textit{dbscan\_expand\_bbox.py} (o \textit{dbscan\_sphere\_bound.py}) e \textit{blensoranalysis} per l'aggiornamento delle bounding box.\\
In \textit{bbtracking.py} vengono aggiornate le bounding box di partenza considerando la pointcloud dello scan successivo. Questi dati, insieme alle pointcloud degli scan successivi, sono ottentuti grazie ai metodi del modulo \textit{blensoranalysis}. Se durante il procedimento di aggiornamento delle bounding box una di esse contiene una porzione di cluster che non è abbastanza grande, allora viene eliminata e considerata noise.\\
\begin{figure}[H]
	\centering
	\begin{subfigure}{0.45\textwidth}
		\includegraphics[width=\textwidth]{bb}
	\end{subfigure}
	\begin{subfigure}{0.45\textwidth}
		\includegraphics[width=\textwidth]{nobb}
	\end{subfigure}
	\footnotesize
	\caption{Caso di cancellazione della bounding box}
\end{figure}

Se durante l'aggiornamento delle bounding box in un determinato istante si presenta una nuova pointcloud che non è associata a nessuna bounding box, allora verrà applicato il clustering di Open3d che genera una nuova bounding box alla quale verrà associato il colore verde. Se quest'ultima si sovrappone a una bounding box già presente, allora viene rimossa quella con volume minore.\\
\begin{figure}[H]
	\centering
	\begin{subfigure}{0.45\textwidth}
		\includegraphics[width=\textwidth]{overlap_remove_1}
	\end{subfigure}
	\begin{subfigure}{0.45\textwidth}
		\includegraphics[width=\textwidth]{overlap_remove_2}
	\end{subfigure}
	\footnotesize
	\caption{Cancellazione di overlapping tra bounding box}
\end{figure}

A livello visivo, verranno mostrate le seguenti situazioni:
\begin{itemize}
	\item quando si hanno le bounding box dell'istante precedente e la pointcloud dell'istante corrente.
		\begin{figure}[H]
			\centering
			\includegraphics[width=0.5\textwidth]{startingpoint}
			\footnotesize
			\caption{Bounding box iniziali e scan successivo}
		\end{figure}
	\item quando si effettua l'operazione di espasione/trasformazione delle bounding box generando una buonding box ingrandita oppure una sfera.
		\begin{figure}[H]
			\centering
			\begin{subfigure}{0.45\textwidth}
				\includegraphics[width=\textwidth]{bbsExpanded}
			\end{subfigure}
			\begin{subfigure}{0.45\textwidth}
				\includegraphics[width=\textwidth]{bbsToSphere}
			\end{subfigure}
			\footnotesize
			\caption{Espansione di bounding box / creazione di sfere}
		\end{figure}
	\item quando si hanno nuovi cluster e quindi nuove bounding box.
		\begin{figure}[H]
			\centering
			\includegraphics[width=0.5\textwidth]{discoveredClusters}
			\footnotesize
			\caption{Nuovo cluster scoperto e bounding box associata di colore verde}
		\end{figure}
	\item quando ogni pointcloud è stata correttamente clusterizzata e le bounding box sono state aggiornate corettamente.
		\begin{figure}[H]
			\centering
			\includegraphics[width=0.5\textwidth]{perfectCluster}
			\footnotesize
			\caption{Creazione delle bounding box aggiornate}
		\end{figure}
\end{itemize}

\newpage
\chapter{Esiti}
Avendo preso in anlisi una situazione ideale di traffico, e cioè che i veicoli sono abbastanza distanziati da non causare errori nella creazione dei punti core e che non variano di direzione bruscamente, si ha che le bounding box aggiornate sono molto precise e che il tempo di esecuzione è sufficientemente basso.\\
\section{Tempo di esecuzione}
Per poter avere una stima temporale del programma, è stata sfruttata la libreria \textit{time} di Python.\\
Il tempo di esecuzione è stato calcolato tenendo conto del tempo di esecuzione dell'algoritmo di clusterizzazione e di aggiornamento delle bounding box.\\
Per ogni esecuzione è stata calcolata la media temporale di 4 scan eseguiti da 1 a 5 sensori.\\
Ovviamente, maggiore è il numero di sensori e maggiore sarà il tempo di esecuzione richiesto.\\
\begin{tikzpicture}
	\begin{axis}[
		title={Tempo di esecuzione del programma},
		xlabel={Numero di sensori},
		ylabel={Tempo medio (s)},
		xmin=1, xmax=5,
		ymin=0, ymax=0.20,
		xtick={1,2,3,4,5},
		ytick={0,0.05,0.10,0.15,0.20},
		legend pos=north west,
		ymajorgrids=true,
		grid style=dashed,
		width=12cm,
		height=8cm,
		axis xshift= -1 cm,
		axis yshift= -1 cm
		font=\footnotesize
	]
	\addplot[
		color=blue,
		mark=square,
		]
		coordinates {
			(1,0.03155684471130371)(2,0.07429736852645874)(3,0.1106995940208435)(4,0.13018083572387695)(5,0.1386846899986267)
		};
	\addplot[
		color=red,
		mark=square,
		]
		coordinates {
			(1,0.034172677993774415)(2,0.08169527053833008)(3,0.11776213645935059)(4,0.1371220588684082)(5,0.14824330806732178)
		};
%	\addplot[
%		color=green,
%		mark=square,
%		]
%		coordinates {
%			(1,5.535)(2,21.219)(3,53.987)(4,104.081)(5,178.993)
%		};
	\legend{Espansione Bounding Box, Sfera}
	\end{axis}
\end{tikzpicture}\\
Dal grafico si evince che il tempo di esecuzione del programma è dell'ordine dei decimi di secondo, un valore più che accettabile dato che gli scan vengono eseguiti a una frequenza di 10Hz.\\
Come previsto, il tempo di esecuzione aumenta all'aumentare del numero di sensori in quanto vengono rilevati più punti nella point cloud e, di conseguenza, anche più oggetti data la posizione dei sensori nell'incrocio.\\

\section{precisione delle bounding box}
Dato che la frenquenza di aggiornamento dei sensori è tale da far variare di poco lo spostamento degli oggeti, non si noterà grande differenza tra i due metodi di clusterizzazione personalizzata.\\
Questa considerazione può essere più impattante nelle situazioni in cui i veicoli si muovono in modo brusco e in direzioni diverse, in quanto il metodo di espansione delle bounding box potrebbe non essere in grado di reagire adeguatamente. Se si ha invece una sfera come riferimento è più probabile che includa eventuali punti in maniera più corretta.\\

\newpage
\chapter{Conclusioni}
Il progetto ha permesso di analizzare le bounding box di una pointcloud utilizzando due approcci differenti. Nonostante i buoni risultati, l'algoritmo DBSCAN presenta dei limiti. Tali limiti sono dovuti soprattutto al fatto che, anche se abbiamo un insieme di punti già clusterizzati, è necessario che venghino comunque rianalizzati affinché i nuovi punti che dovranno essere inglobati in questi cluster rispettino i concetti di \textit{density-connected} e \textit{core point}, come spiegato nel pargrafo relativo al clustering.\\
Inoltre, essendo una mole di dati elevata, il programma avrebbe giovato da una maggiore potenza di calcolo, soprattutto da parte di una scheda grafica. Non a caso vengono sfruttate soprattutto le schede video per elaborare grandi quantità di dati in tempo reale nella guida autonoma.\\
\\
Delle possibili migliorie nelle tecniche di clustering possono essere la riduzione del tempo di esecuzione e un sistema di spostamento delle bounding box in base a come si muovono nello spazio.\\
Per quanto riguarda la prima considerazione, essendo dipendente soprattuto dalle elevate dimensioni dei file .csv dai quali il programma importa le pointcloud (hanno un peso di circa 11MB), questa procedura verrebbe eseguita istantaneamente da sensori fisici, senza la necessità di dover leggere delle informazioni da dei file. Quindi questo tipo di problema non si porrebbe più. Ma comunque, trovandosi in un ambiente di testing, può sempre tornare utile e una possibile modifica sarebbe quella di pre-filtrare le coordinate dei punti che rappresentano il terreno.\\
Per la seconda considerazione invece si potrebbe implementare un sistema di spostamento delle bounding box in base alla velocità dei veicoli e alla loro direzione di marcia. Tali informazioni possono essere estrapolate considerando l'andamento dei veicoli di cui si sta tenendo traccia nei vari scan.\\
Questa integrazione è utile soprattutto nelle possbili situazioni di traffico intenso, dove i veicoli sono a stretto contatto. Il rischio che si va a correre in questi scenari, dato che tutti i punti all'interno della bounding box vengono labelizzati a priori, è quello di considerare cluster anche parte di un oggetto che si trova internamente alla bounding box, ma che in realtà non vi dovrebbe appartenere in quanto parte di un altro oggetto.\\

\newpage
\chapter*{Grazie di Cuore}
{\ssfamily\textit{
Non ho mai avuto un percorso lineare nella mia vita a causa di una infanzia difficile. Spesso mi sono trovato ad affrontare situazioni inaspettate, in cui non avrei mai pensato di trovarmi. In quei momenti mi sentivo perso e non sapevo come uscirne. Durante il corso degli anni ho avuto la fortuna di avere amici e parenti che mi hanno aiutato ad affrontarle.\\
\\
In particolare, volevo ringraziare i miei cugini Alessandro e Giorgia, a cui devo tutto, per avermi reso indipendente e per avermi trattato come se fossi un loro fratello. Non dimenticherò mai ciò che mi avete insegnato, che è ormai alla base della mia personalità. Spero di rivedervi presto, mi mancate un sacco.\\
\\
Un ringraziamento va anche ai miei genitori, che nel loro piccolo sono sempre riusciti a non farmi mai mancare nulla. Ringrazio anche zia Mimma, zia Paola e Sonya che non mi hanno mai fatto mancare l'affetto di una famiglia.\\
\\
Un grazie a Ivan e Michele con cui ho sempre potuto parlare liberamente di tutto e mi sopportano ormai da anni.\\
\\
Grazie a Silenia, Lito e Christian che mi hanno aiutato a maturare ulteriormente senza mai chiedere nulla in cambio. Spero un giorno di poter ricambiare.\\
\\
Grazie a Francesco, Roberto e Matteo che mi hanno sempre tirato su di morale.
}}

\newpage
\begin{thebibliography}{9}
	\bibitem{LiDAR} \href{https://it.wikipedia.org/wiki/Lidar}{Wikipedia. Lidar. 2024.}
	\bibitem{Open3D} \href{http://www.open3d.org/}{Open3D: A Modern Library for 3D Data Processing.}
	\bibitem{DBSCAN} \href{https://www.geeksforgeeks.org/dbscan-clustering-in-ml-density-based-clustering/}{Geek for Geeks. DBSCAN Clustering in ML. 2024.}
\end{thebibliography}

\end{document}
